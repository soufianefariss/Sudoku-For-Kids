\documentclass{article}
\usepackage[utf8]{inputenc}
\usepackage{xcolor}
\usepackage{listings}
\usepackage{hyperref}
\usepackage[
    type={CC},
    modifier={by-nc-sa},
    version={3.0},
]{doclicense}

\usepackage{scrlayer}
\DeclareNewLayer[
    foreground,
    %textarea,% use only the textarea
    contents=\blankpage
  ]{blankpage.fg}
\DeclarePageStyleByLayers{blank}{blankpage.fg}

\usepackage{scrlayer}
\usepackage{mwe}
\usepackage{multicol}

\makeatletter
    \def\cleardoublepage{\clearpage%
        \if@twoside
            \ifodd\c@page\else
                \vspace*{\fill}
                \hfill
                \begin{center}
                This page intentionally left blank.
                \end{center}
                \vspace{\fill}
                \thispagestyle{empty}
                \newpage
                \if@twocolumn\hbox{}\newpage\fi
            \fi
        \fi
    }
\makeatother

\usepackage[english]{babel}

\usepackage{lipsum}

\definecolor{mGreen}{rgb}{0,0.6,0}
\definecolor{mGray}{rgb}{0.5,0.5,0.5}
\definecolor{mPurple}{rgb}{0.58,0,0.82}
\definecolor{backgroundColour}{rgb}{0.95,0.95,0.92}

\lstdefinestyle{CStyle}{
    backgroundcolor=\color{backgroundColour},   
    commentstyle=\color{mGreen},
    keywordstyle=\color{magenta},
    numberstyle=\tiny\color{mGray},
    stringstyle=\color{mPurple},
    basicstyle=\footnotesize,
    breakatwhitespace=false,         
    breaklines=true,                 
    captionpos=b,                    
    keepspaces=true,                 
    numbers=left,                    
    numbersep=5pt,                  
    showspaces=false,                
    showstringspaces=false,
    showtabs=false,                  
    tabsize=2,
    language=C
}
\usepackage[utf8]{inputenc}
\usepackage{natbib}
\usepackage{graphicx}
 
\usepackage{epigraph}
\begin{document}

\textbf{Remerciements} 

Nous adressons nos sincères remerciements à tous les professeurs, intervenants et toutes les personnes qui par leurs paroles, leurs écrits, leurs conseils et leurs critiques, ont guidé nos réflexions et ont accepté de nous rencontrer et de répondre à nos questions durant nos recherches.
 
Nous remercions nos très chers parents, qui ont toujours été là pour nous.
 
\epigraph{Vous avez tout sacrifié pour vos enfants sans ménager votre santé ni vos efforts. Vous nous avez offert un merveilleux modèle de travail acharné et de persévérance. Nous sommes redevables à une éducation dont sommes fiers}{\textit{Soufiane FARISS \\ Ismail FATRI}}

\textbf{Encadrant: Mr. A. EL FAKER.} 
\newpage


\begin{center}

  \uline{\bfseries À Propos Des Auteurs}\\

\end{center}

\textbf{Soufiane FARISS}
\epigraph{
A nerd trying to make a difference in the world.\\ 
Currently studying basic InfoSec and Software engineering at The National School For Computer Science (ENSIAS) of Rabat, Morocco.


}{\textit{Soufiane FARISS \\Option: SSI}}



\textbf{Ismail FATRI}
\epigraph{
Élève-ingénieur à l'école nationale ENSIAS passionné par le monde de la sécurite numérique.
}{\textit{Ismail FATRI \\Option: SSI}}

\newpage
\vspace*{\fill}
                \hfill
                \begin{center}
                Cette page a été intentionnellement laissée vierge.
                \end{center}
                \vspace{\fill}
                \thispagestyle{empty}

\newpage
\tableofcontents
\newpage
\title{C-doku pour enfants}

\author{
  Soufiane FARISS\\
  \texttt{soufiane.fariss@um5s.net.ma}
  \\Université Mohammed V\\ENSIAS
  \and
  Ismail FATRI\\
  \texttt{ismailfatri@gmail.com}
  \\Université Mohammed V\\ENSIAS
}

\date{Janvier 2019}
\maketitle


\section{Introduction}
Le Sudoku est un jeu de réflexion créé en 1979 par un Américain. Ce jeu consiste à remplir une grille de 4x4 par des icônes en respectant certaines contraintes.
La grille de jeu est un carré de 4 cases de côté, subdivisé en autant de carrés identiques, appelés régions. On donne à la disposition du joueur 4 différentes icônes.
La règle du jeu est simple : chaque ligne, colonne et région ne doit contenir qu'une seule fois l’icône en question. Formulé autrement, chacun de ces ensembles doit contenir tous les icônes sans aucune répétition.

\begin{figure}[h!]
\centering
\includegraphics[scale=0.5]{sudoku_board}
\caption{sudoku 4x4}
\label{fig:sudoku_board}
\end{figure}

\section{l'objectif du projet (Cahier des charges)}

Il faut développer une application en C pour le jeu du sudoku
(16 cases). Au départ des cases sont pré-remplies et ne peuvent être changées.

Les 4 régions sont repérables par la couleur des briques. Chaque élément (étoile, cercle, triangle, carré) doit être présent une et une seule fois sur :

	• chaque ligne ;
    
	• chaque colonne ;
    
	• chaque région ;
    

\noindent En plus de pouvoir jouer, on devra pouvoir :

	• Recommencer à tout moment en cliquant sur le bouton "recommencer"
    
    • Créer et enregistrer les joueurs et leurs caractéristiques (nom et score) sur 		fichier
    
	• Permettre le chargement d’un jeu sauvegardé auparavant.
    
	• Donner la liste des dix meilleurs scores
    
    
\noindent Il est également recommandé de considérer :

	• La sécurité des accès à l’aide de mots de passe
    
	• La réaliser d’une interface de jeu conviviale

\section{Fonctionnement}
Au début, vous possédez deux choix :

    • soit s’inscrire si c’est votre première partie dans ce jeu :     vous allez entrer votre identifiant puis vous choisissez un mot de passe.
    
    • Soit se connecter si vous y avez déjà joué : vous allez vous connectez à l’aide de l’identifiant et mot de passe déjà fournis.
    \newline
    
\noindent Ensuite, vous allez être redirigé vers l’interface de l’utilisateur qui vous donne les possibilités suivantes:

    • Lancer une nouvelle partie en générant une grille qui est pré remplie par le biais du bouton \textbf{\textit{nouveau}}.
    Une fois que vous avez remplie les cases vides vous devez cliquer sur le bouton \textbf{\textit{soumettre}} pour vérifier votre résultat et enregistrer votre score.
    
    • Charger une partie sauvegardée auparavant à l’aide du bouton \textbf{\textit{charger}} 
    
    • Afficher le classement des 10 meilleurs joueurs en appuyant sur le bouton \textbf{\textit{TOP 10}}.
    
    • Quitter le jeu en appuyant sur le bouton \textbf{\textit{Quitter}}.
    
\section{Outils et techniques de developpement}
Pour la réalisation du travail, nous avons utilisé le langage C dans l'éditeur gedit (UNIX), et afin d’obtenir une interface graphique, nous avons utilisé GTK+ qui est un ensemble de bibliothèques logicielles permettant de réaliser des interfaces graphiques. GTK+ est écrit en langage C et utilise pourtant le paradigme de la programmation orientée objet. Il est également possible d'utiliser GTK+ dans de nombreux autres langages de programmations: Perl, C++, Java, JavaScript, Python, PHP…

\section{Scenario d’execution}
\subsection{Inscription}
\textit{Il est nécessaire de s’inscrire pour pouvoir lancer une partie}.

\begin{figure}[h!]
\centering
\includegraphics[scale=0.5]{login.png}
\caption{Log in}
\label{fig:login}
\end{figure}

\noindent
\newline

\subsection{Authentification}
\textit{S’il décide de s’inscrire, il doit entrer son identifiant et mot de passe choisis}.
\newline
\begin{figure}[h!]
\centering
\includegraphics[scale=0.5]{auth.png}
\caption{Authentification}
\label{fig:sudoku_board}
\end{figure}

\textit{
Une fois inscrit, il pourra se connecter en utilisant l’identifiant et mot de passe choisis auparavant}.
\newline

\subsection{Game on!}
Dès qu’il accède à son compte, un menu sera affiché qui demande au utilisateur soit jouer une partie, soit sauvegarder sa partie, soit se quitter.

Une fois la partie commencée en cliquant sur le bouton nouveau:L’utilisateur aura le choix soit de jouer, soit de sauvegarder l’évolution de sa partie, soit charger une partie sauvegardée, soit quitter.

\subsubsection{Button: \textit{New}}
Cette fonction génère une carte de sudoku aléatoire à l’aide du fichier \textit{generator.exe} et l’enregistre dans \textit{savedsudoku.txt}.
\begin{lstlisting}[style=CStyle]

static void new( GtkWidget *widget, gpointer data)
{
	FILE *sudoku_file = NULL;
	int i,j;
	char n[2] = { 0, '\0' };
	char c;

	GdkColor color;
	gdk_color_parse( "#eeeeee", &color );
	
	system("generator"); // Cet fonction execute le fichier generator.exe 
	
	sudoku_file = fopen(CURRENT_PUZZLE_PATH, "r" );
	
	// Remplissage de la grille
	for( i = 0; i < 4; i++ ) {
		for( j = 0; j < 4; j++ ) {
			c = fgetc(sudoku_file);
			if( c != '.' )
			{
				n[0] = c;
				gtk_widget_set_sensitive(sudokuw[i][j], FALSE);
			}
			else
			{
				n[0] = '\0';
				gtk_widget_set_sensitive(sudokuw[i][j], TRUE);
				gtk_widget_modify_bg( sudokuw[i][j], GTK_STATE_NORMAL, &color );
			}
			gtk_button_set_label( GTK_BUTTON(sudokuw[i][j]), n );
		}
	}
	fclose(sudoku_file);
}
\end{lstlisting}

\subsubsection{Button: \textit{Save}}
Cette fonction enregistre la progression du jeu dans le fichier \textit{savedsudoku.txt}.
\begin{lstlisting}[style=CStyle]
static void save( GtkWidget *widget, gpointer data ) {
	FILE *savefile = NULL;
	int i,  j;
	savefile = fopen( SAVE_PATH, "w" );
	if (!savefile)
		return;

	//fprintf( savefile, "%s\n0", sudoku_files[current_sudoku] );

	for( i=0; i<4; i++ ) {
		for( j=0; j<4; j++ ) {
				if( strcmp( "", gtk_button_get_label(GTK_BUTTON(sudokuw[i][j])) ) != 0 )	// if user wrote something
				{
					fprintf( savefile, "%s", gtk_button_get_label( GTK_BUTTON(sudokuw[i][j])) );
					continue;
				}
				fprintf( savefile, "." );
		}
	}
	fclose( savefile );
}
\end{lstlisting}

\subsubsection{Button: \textit{Sumbit}}
Ce button vérifie votre résultat et enregistre votre score.
\begin{lstlisting}[style=CStyle]
static void submit(GtkWidget *widget, gpointer data ) {
	srand(time(NULL));
	char board[4][4], *sound_effects[6] = {
							"canberra-gtk-play -f ~/myProject/ressources/sounds/no-1.wav",
							"canberra-gtk-play -f ~/myProject/ressources/sounds/no-2.wav", 
							"canberra-gtk-play -f ~/myProject/ressources/sounds/no-3.wav", 
							"canberra-gtk-play -f ~/myProject/ressources/sounds/no-4.wav", 
							"canberra-gtk-play -f ~/myProject/ressources/sounds/no-5.wav", 
							"canberra-gtk-play -f ~/myProject/ressources/sounds/no-6.wav"
							};
							
							
	for(int i=0; i<4; i++) {
		for(int j=0; j<4; j++ ) {
				const gchar *c = gtk_button_get_label( GTK_BUTTON(sudokuw[i][j]));
				if (*c == 0) {
					char r = rand() % 6;
					system(sound_effects[r]);
					return;
				}
				board[i][j] = *c;
		}
	}
	
	for (int row = 0; row < 4; row++) {
		for (int i = 0; i < 4 - 1; i++) {
			for (int j = i + 1; j < 4; j++) {
				if (board[row][i] == board[row][j]) {
				    char r = rand() % 6;
					system(sound_effects[r]);
					return; // 0 means a duplicate is found, quitting..
				}
			}
		}
	}
	
	int transposed[4][4];
    for (int i = 0; i < 4; i++) 
        for (int j = 0; j < 4; j++) 
            transposed[i][j] = board[j][i];
	
	// now, check columns... 
	for (int col = 0; col < 4; col++) {
		for (int i = 0; i < 4 - 1; i++) {
			for (int j = i + 1; j < 4; j++) {
				if (transposed[col][i] == transposed[col][j]) {
				    char r = rand() % 6;
					system(sound_effects[r]);
					return; // 0 means a duplicate is found, quitting..
				}
			}
		}
	}
	system("canberra-gtk-play -f ~/myProject/ressources/sounds/Ta-Da.wav");
	return; // means good board!
}

\end{lstlisting}

\subsubsection{Button: \textit{TOP 10}}
Ce button afficher les TOP 10 scores.
\begin{lstlisting}[style=CStyle]


// TODO: A COMPLETER


\end{lstlisting}

\subsubsection{Button: \textit{Quitter}}
Ce bouton permet de quitter le jeu.
\begin{lstlisting}[style=CStyle]
// quit
static void quit( GtkWidget *widget, gpointer data )
{
  	gtk_main_quit();
}
\end{lstlisting}

\subsubsection{Chronomètre}
Cette section présente diverses fonctions utilisées pour implémenter un chronomètre dans GTK +. Ce bloc de code consiste à afficher un chronomètre à l'utilisateur avec possibilité d'arrêt ou de réinitialisation.

\begin{lstlisting}[style=CStyle]
/* Determines if to continue the timer or not */
static gboolean continue_timer = FALSE;

/* Determines if the timer has started */
static gboolean start_timer = FALSE;

/* Display seconds expired */
static int sec_expired = 0;

static gboolean _label_update(gpointer data) {
	GtkLabel *label = (GtkLabel*)data;
	char buf[256];
	memset(&buf, 0x0, 256);
	snprintf(buf, 255, "<span font_desc=\"Ubuntu Mono 14\"> Time elapsed: %d secs </span>", sec_expired++);
	gtk_label_set_markup(GTK_LABEL(label), buf);
	return continue_timer;
}

static void _start_timer (GtkWidget *button, gpointer data) {
    (void) button; /*Avoid compiler warnings*/
    GtkWidget *label = data;
    if(!start_timer) {
        g_timeout_add_seconds(1, _label_update, label);
        start_timer = TRUE;
        continue_timer = TRUE;
    }
}

static void _pause_resume_timer (GtkWidget *button, gpointer data)
{
    (void)button;
    if(start_timer)
    {
        GtkWidget *label = data;
        continue_timer = !continue_timer;
        if(continue_timer)
        {
            g_timeout_add_seconds(1, _label_update, label);
        }
        else
        {
            /*Decrementing because timer will be hit one more time before expiring*/
            sec_expired--;
        }
    }
}
\end{lstlisting}

\begin{figure}[h!]
\centering
\includegraphics[scale=0.5]{timer.png}
\caption{Chronomètre}
\label{fig:timer}
\end{figure}

\section{Le générateur}
Ce générateur génère un tableau aléatoire en utilisant un fichier contenant 256 formations possibles. Nous prenons d’abord un état de départ aléatoire (exemples 2, 4, 4, 1), puis nous remplissons une variable appelée init\_boad et nous utilisons quelques fonctions afin de générer un tableau de départ valide. 
\textit{(Voir generator.c)}

Ce générateur est u \underline {pseudo-générateur} simplifié et peut générer des états de départ \textbf{\textit{invalides}}.

\section{Comment Exécuter?}
\noindent Pour compiler le projet, exécuter la commande suivante:
\begin{lstlisting}[language=bash]
  $ make sudoku
\end{lstlisting}

\noindent Pour supprimer le projet, exécuter la commande suivante:
\begin{lstlisting}[language=bash]
  $ make clean
\end{lstlisting}

\begin{center}
\centering 
\includegraphics[scale=0.6]{makefile.png}
\captionof{Figure 6}{ : Makefile}
\end{center}

\newpage

\section{Aspect général}
\indent Application finale:
\begin{center}
\centering 
\includegraphics[scale=0.5]{overall.png}
\captionof{Figure 7:}{ : Application Finale}
\end{center}




\section{Conclusion}
A travers ce projet, nous avons pu comprendre et expérimenter les différentes étapes de la conception d'un jeu pour enfants qu’est le SUDOKU. 
De plus la réalisation du programme nous a permis non seulement d'approfondir nos connaissances acquises depuis le début de notre formation, mais aussi de nous familiariser avec la bibliothèque utilisée GTK.
En plus d’être un projet pédagogique il est aussi ludique et nous a donné beaucoup de liberté dans le code et dans la conception.

\section{License}
Copyright (c) 2019 github.com/soufianefariss
\doclicenseThis

\end{document}
